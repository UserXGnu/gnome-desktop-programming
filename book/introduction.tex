\chapter{Introduction}

Your presence here is important. You have made the choice to remove the magic
curtains from your computing experience by learning the low-level computer
programming language, C. The history of C is long, and can be much better
explained in other mediums. Many changes to the language have occurred over
the past 40 years. This book aims to teach you the C programming language as
it is today, only explaining ancient portions when it provides a better
understanding of the now.

This book covers the C programming languages, from the very basics to advanced
concepts. More than that, it is a companion for your journey in Linux and the
GNOME Desktop. Once we have a grasp of the language we will learn the
components available to us and how to use them effectively. We will cover the
concepts behind the choices. We will take note of designs and concepts have
not served us well and learn from them.

The next phase of our journey is one of preparation.


\section{Terminal}

Being comfortable at a terminal is an essential skill for a programmer. It
provides a no-frills, get-things-done, interface to the computers operating
system. Most of this book will involve the use of a terminal.

You should be familiar with changing directories, typing commands, and
analyzing the result of those commands.

Many GNOME users are familiar with \ident{gnome-terminal}, which is what your
author is using. However, \ident{xterm} is another common, high-quality
terminal. \ident{gnome-terminal} should have been installed as part of your
GNOME installation.


\section{Preparing the System}

To write software for a particular platform, such as GNOME, you need an SDK
(Software Development Kit). An SDK contains information about the platform. It
typically includes a compiler, headers, and other associated files. Sometimes,
people call this a "toolchain".

A compiler is a program that takes source code, and translates it into a
language that the computer knows. If it cannot understand it, it will not be
able to translate it and the compiler will try to give us a hint as to what
went wrong.

"Headers" contain information about the programs on our computer.  It gives us
insight into how the programs were written. When we want to make use of
existing software on the system, we use that software's "headers" to prime the
compiler as to how it works. The compiler than then takes our source code and
with what it learned from the headers to translate our program into something
the machine can execute.


\section{Installing the SDK}

GNOME contains many pieces of software written by many different groups of
people. We need a compiler, headers, and associated tools for building
software. To get those, we will use our Linux distributions package manager to
install the required components.

It is important to be comfortable with a terminal for software development. A
terminal is essential to the process of software development. So get those
terminals going! I use gnome-terminal, but everyone has there favorite; xterm
for example.

We also need a text editor to use to write our code. Every programmer has their
favorite. I, love using VIM in a terminal. Many C programmers love Emacs. And
some GNOME programmers are perfectly happy in Gedit. For this book, I will
assume you are using Gedit, because the VIM and Emacs users probably know their
way around their editor just fine.


\subsection{Fedora 17}

\begin{Verbatim}
    $ sudo yum install gtk3-devel clutter-devel \
      gobject-introspection-devel webkitgtk3-devel \
      glibc-devel make automake autoconf pkg-config gedit
\end{Verbatim}


\subsection{Ubuntu 12.04}

\begin{Verbatim}
    $ sudo apt-get install build-essential libgtk3.0-dev \
      pkg-config gedit
\end{Verbatim}


\subsection{Arch Linux}
\begin{Verbatim}
    $ sudo pacman -S base-devel clutter-gtk gtk gedit
\end{Verbatim}

\subsection{Debian Wheezy}

\begin{Verbatim}
    $ sudo apt-get install build-essential libgtk3.0-dev \
      pkg-config gedit
\end{Verbatim}


\section{Kicking the Tires}

Let's double-check that we got everything installed propertly so we don't have
an unexpected problems later as we perform the exercises in this book.  We will
use the \ident{pkg-config} program to check what versions of programs we just
installed. \ident{pkg-config} is a program that tells us information about
various components of our installed SDK.

\begin{Verbatim}
    $ pkg-config --modversion gobject-2.0
    2.33.14

    $ pkg-config --modversion gtk+-3.0
    3.5.18
\end{Verbatim}

If the above commands give you an error about the package not being found, then
you still need to install the development (-dev) packages for the given
libraries.
