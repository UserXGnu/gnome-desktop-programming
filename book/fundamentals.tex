\chapter{Fundamentals}

This chapter will get you familiar with the C syntax, terminology, and
process for reading and writing C code.

Like any sport, the fundamentals of C are critical to your success. If you have
a strong understanding of these, no challenge will be unsurmountable with the
right amount of patience and determination.

Today, you start on those fundamentals. We will get things going by looking at
some C code, and then familiarizing ourselves with the anatomy.


\begin{code}{figure1.c}
\input{fundamentals1}
\end{code}


TODO: Lots of explaination on the anatomy of figure 1.


\section{Arithmetic}

Every computer language, at some point, is working with numbers and basic
arithmetic. Thankfully, you don't have to be good at mathematics to be a good
programmer. But we should have a knowledge of the basics. Performing arithmetic
in the C programming language should feel similar to what you already do when
writing equations on a piece of paper.

Addition is performed with the \ident{+} operator and subtraction with the
\ident{-} operator. Multiplication is performed with the \ident{*} operator and
division with the \ident{/} operator. Determining the remainder of division can
be found with the \ident{\%}, "modulus", operator.

TODO: examples of the arithmetic.

\subsection{Order of Operations}

Just like you may have learned in beginning algebra, parenthesis can be used
to alter the order of operations.

TODO: explain the order of operations.

\section{Binary}

\section{Types}
