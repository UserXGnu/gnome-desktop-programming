\chapter{If/Else If/Else}

\ident{if} blocks are an essential part of programming in any language. They
allow you to perform a series of operations only when a certain
\ident{condition} occurs. Let's take a look at an example and learn by
dissecting it into its component parts.

\begin{code}{ifelse1.c}
#include <stdio.h>

void main ()
{
    int a;

    printf ("Please enter an integer between 1 and 100:  ");
    scanf ("%d", &a);

    if (a < 1) {
        printf ("%d is less-than 1\n", a);
    }

    if (a > 100) {
        printf ("%d is greater-than 100\n", a);
    }

    if ((a >= 1) && (a <= 100)) {
        printf ("%d / 13 is %d\n", a, a / 13);
    }
}
\end{code}

The example above provides us 3 separate examples of an \ident{if} block that
we can analyze. The first is begins on line 10, the second on line 14, and third
on line 18.

An \ident{if} block follows the following syntax\footnote{I have chosen to
illustrate this example using curly braces. While you can perform single
operations without curly braces, it often results in difficult to find bugs.}.

\begin{Verbatim}
if (expression) {
    statement
}
\end{Verbatim}

Lets apply this knowledge to determine the expression and statement of the
\ident{if} blocks in the above example.

In the first example, we have:

\begin{code}{}
if (a < 1) {
    printf ("%d is < 1\n", a);
}
\end{code}

In this example the expression would be \verb|a < 1| and the statement
would be \verb|printf ("%d is < 1\n", a);|. The \verb|<| operator asks the
question \verb|less than|. So if we were to read this expression in English,
we might say "if \verb|a| is \verb|less than| \verb|1|".

In the second example, we have:

\begin{code}{}
if (a > 100) {
    printf ("%d is > 100\n", a);
}
\end{code}

In this example the expression would be \verb|a > 100| and the statement
would be \verb|printf ("%d is > 100\n", a);|. The \verb|>| operator asks
the question \verb|greater than|. So if we were to read this expression in
English, we might say "if \verb|a| is \verb|greater than| \verb|100|".

In the third example, we have:

\begin{code}{}
if ((a >= 1) && (a <= 100)) {
    printf ("%d / 13 is %d\n", a, a / 13);
}
\end{code}

This example is much more complex than the previous two.  We have four new
concepts in this single example. The first two are new operators, \verb|<=| and
\verb|>=|, which mean \verb|less than or equal to| and \verb|greater than or equal to|
respectively. Another new operator, \verb|&&|, is the boolean
\verb|and| operator. The fourth new concept in this example is the use of
parenthesis similar to what you may have learned in basic algebra.  Parenthesis
can be used to make expressions more obvious to the reader as well as ensure
what you meant is exactly what the compiler thinks you meant. Your author has
seen many bugs that could have been easily prevented by using parenthesis
liberally.

If we were to write this expression in English, it might read like "if \verb|a|
is greater than or equal to \verb|1|, and, \verb|a| is less than or equal to
\verb|100|".

TODO:

\begin{enumerate}
\item Show the part that is the conditional so that we can reuse that
      knowledge with loops.
\item Make sure it is clear you can have an if without the others.
\item Cover boolean operators.
\item Cover implicit "truthiness" when non-zero.
\item Introduce TRUE/FALSE from glib?
\end{enumerate}

\begin{code}{ifelse1.c}
#include <stdio.h>

void main ()
{
    int a;

    print ("Please enter an integer:  ");
    scanf ("%d", &a);

    if (a > 100) {
        printf ("%d is > 100\n", a);
    } else if (a < 100) {
        printf ("%d is < 100\n", a);
    } else {
        printf ("%d is == 100\n", a);
    }
}
\end{code}

\begin{Terminal}
gcc ifelse1.c
./a.out
100 is == 100
\end{Terminal}


TODO:

Truthiness of all non-zero expressions.

\begin{tabular}{l | r}
1 & TRUE \\
0 & FALSE \\
150 & TRUE \\
-1 & TRUE
\end{tabular}
