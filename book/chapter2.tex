\chapter{Decimal Numbers}

In the last chapter, we learned about integers. They are indeed handy, but
provide little utility when needing to work with numbers where a decimal point
is required.

The C language provides another type to indicate we want to store numbers that
contain a decimal point. It is called a \verb|double|\footnote{Another type
called float also supports decimal precision. It will be covered in a later
chapter.}.

Let's take a look at a short example comparing the storage of a decimal number
in both an \verb|integer| and a \verb|double|.

\begin{code}{decimal1.c}
#include <stdio.h>

void main ()
{
    int a = 123.45;
    double b = 123.45;

    printf ("%d %lf\n", a, b);
}
\end{code}

\begin{Terminal}
gcc decimal1.c
./a.out
123 123.45
\end{Terminal}


TODO
\begin{enumerate}
\item numbers with decimals
\item printf with \%lf
\item floating point and imprecise precision
\end{enumerate}


\section{Student Exercise}

Use what you have learned this chpater to print the division of two decimal
numbers. Use the numbers provided in the example below to test your
implementation. Error handling is not required in this exercise. Chapter 1 used
\verb|%d| inside of the call to \verb|printf|. In this exercise, use \verb|%lf|
instead.

\begin{code}{student2.c}
#include <stdio.h>

void main ()
{
    double a;
    double b;

    printf ("Enter two decimal numbers to multiply: ");
    scanf ("%lf %lf", &a, &b);

    /* TODO: Print the division of a and b. */
}
\end{code}

When complete, the output should look like the following.

\begin{Terminal}
gcc student2.c
./a.out
Enter two decimal numbers to multiply: 12.3 45.6
12.3 / 45.6 = 0.269737
./a.out
Enter two decimal numbers to multiply: 71 873646.999
71.000000 / 873646.999000 = 0.000081
\end{Terminal}

\subsection{Extra Credit}

Some division is simply not possible. What possible input could a user provide
that would result in this problem?
