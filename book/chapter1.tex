\chapter{The Basics}

The C programming language often seems complex and mysterious to those who have
not yet been emersed. We have all heard the horror stories of how difficult
pointers can be. While it is indeed a difficult subject, it is perfectly within
your reach.

The focus of this chapter is to familiarize yourself with the very basics of
the language. You will write your first C program as we reason how to express
ourselves in this unfamiliar place.

Occasionally, we will gloss over a few details so that we can broaden our
understanding. Future chapters may feel repetitious as we reinforce previous
teachings in the light of new understanding.

Lets take a quick look at the stereotypical first program. The venerable
"hello, world" exercise. Create a new text file with the following C code and
save it as \file{hello.c}.

\begin{code}{hello.c}
#include <stdio.h>

void main ()
{
    printf ("Hello, World!\n");
}
\end{code}

Navigate using your terminal to the directory where you saved the file. Type in
the following command. It will compile the program into a format that the
computer knows how to execute.

\begin{Verbatim}
gcc hello.c
\end{Verbatim}

If everything worked, the \ident{gcc} command will exit silently and a new
executable file \file{a.out} will be found in your current directory. If there
was an error, carefully check that each character in \file{hello.c} matches the
C code above and try again.

We can execute the program from the terminal by prefixing \verb|./| to the
filename to tell the terminal to execute a file within the current directory.

\begin{Verbatim}
./a.out
Hello, World!
\end{Verbatim}

There you have it, your very first C program!

Lets now analyze the anatomy of this simple C program. The very first line,
\verb|#include <stdio.h>| tells the compiler that we would like to use the
\ident{stdio} library. A library is a set of useful, reusable routines. In this
case, the \ident{stdio} library contains routines for interfacing with standard
input and output. Run \verb|man stdio| in your terminal for more information on
the \ident{stdio} library.

The line following our include is simply an empty line. Empty lines do not
effect the program. In some languages, this does matter. However, C is not one
of those languages. You may use empty lines liberally to make your code more
readable.

Next, we have \verb|void main ()|. This line tells the compiler that we have a
function named \ident{main}. The first word, \ident{void} tells the compiler
that the function will not be returning a value. Lastly, \ident{()} denotes the
argument list for the function. Our \ident{main} function takes no arguments
and so it has none specified\footnote[1]{This isn't technically true, but that
will be covered in a much later chapter.}. Do not worry if this seems
mysterious, we will cover it in great detail later on.

After the line defining our function \ident{main}, we have a \verb|{|. This
denotes the beginning of the body of the function. There is a coordinating
\verb|}| at the end of the function. Inside of these curly braces is the crux
of our program.

\verb|printf ("Hello, World!\n");| is a function call to the \ident{printf}
function (part of the \ident{stdio} library). You can tell it is a function
call because the function name is followed by a parenthesized argument list.
In this case, our function call has a single argument,
\verb|"Hello, World!\n"|. Information can be shared between functions in
a couple different ways. This way, through the argument list, is the most
common.

Our \verb|"Hello, World!\n"| argument is what we call a "string". A string
is a sequence of characters. At the end of the string you will see \verb|\n|.
This means that the string should contain a line break at the end of it.

We denote the end of a \ident{statement} in C with a semicolon \ident{;}.
What exactly a \ident{statement} is will be vague for the time being, but it
will become obvious as we continue.

\section{Integers}

Writing a Hello World program is always fun in a new language, but it wont
exactly win us any awards. Lets try to print out some interesting numbers
using our program as a base.

Lets create a new text file with the following and save it as \file{answer.c}.

\begin{code}{answer.c}
#include <stdio.h>

void main ()
{
    int answer = 42;
    printf ("The answer is %d, but what is the question?\n", answer);
}
\end{code}

Once again, lets use \ident{gcc} to compile the program. You will get very
used to this in no time!

\begin{Verbatim}
gcc answer.c
\end{Verbatim}

And again, lets run it.

\begin{Verbatim}
./a.out
The answer is 42, but what is the question?
\end{Verbatim}

Since we have covered the anatomy of the \ident{main} function earlier in this
chapter, lets skip to the two lines inside the body of the function.

The first line, \verb|int again = 42;|. This line declares a new variable named
\verb|again| of the type \verb|int|. \verb|int| is short for \verb|integer|.
You might remember that an integer is a "whole" number, such as 1, 2, or -20.
There are no decimal points in an integer. So if you tried to store the number
\verb|12.5| in an integer, it would simply be 12. The second half of this
statement, \verb|= 42| initialized the variable \verb|again| to the value
\verb|42|.

The second line should look familiar. It again is calling the \verb|printf|
function. However, this time there are two arguments. We will cover this in
more detail in a later chapter, but there are a few important things to take
from this example. First, notice the comma \verb|,| used to separate the
two arguments provided to \verb|printf|. In C, arguments are separated by
commas. Also, note the \verb|%d| inside of the first parameter. This is a
magic value that \verb|printf| will replace with the value of the second
parameter.

\section{Multiplication}

Integers don't seem to be all that useful when used by themselves like this.
Lets take a look how multiplication of integers works with an example. Type
the following code into a new text file named \file{multiply.c}.

\begin{code}{multiply.c}
#include <stdio.h>

void main ()
{
    int a = 42;
    int b = 32;

    printf ("%d * %d = %d\n", a, b, a * b);
}
\end{code}

And, again, we compile and run it.

\begin{Verbatim}
gcc multiply.c
./a.out
42 * 32 = 1344
\end{Verbatim}

In this example, we have two variables defined, \verb|a| and \verb|b|
initialized with the values 42 and 32 respectively. Following that, like before
we call the function \verb|printf|. This time, there are four arguments. You can
see that there are three \verb|%d|s used in the first argument to \verb|printf|.
One for each of the arguments following it. The following two arguments,
unsurprisingly, cause the values 42 and 32 to be placed into the printed line.
However, the last argument, \verb|a * b| performs the multiplication of the
variables \verb|a| and \verb|b| as denoted by the multiplication operator
\verb|*|. The result of this multiplication is placed in the location of the
last \verb|%d| when printing to the screen.

\section{Student Exercise}

TODO: Explain the exercise.
