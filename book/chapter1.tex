\chapter{The Basics}

The C programming language often seems complex and mysterious to those who have
not yet been emersed. Many of us have heard horror stories about how difficult
pointers can be. While it is indeed a difficult subject, it is perfectly within
your reach.

The focus of this chapter is to familiarize yourself with the very basics of
the language. You will write your first C program and we will reason how to
express ourselves in this unfamiliar place.

Occasionally, we will gloss over certain details so that we can broaden our
understanding. Future chapters may feel repetitious as we reinforce previous
teachings in the light of new understanding.

Lets take a quick look at the stereotypical first program. The venerable
"hello world" exercise. Create a new text file with the following C code and
save it as \file{hello.c}.

\begin{code}{hello.c}
#include <stdio.h>

main ()
{
    printf ("Hello, World!\n");
}
\end{code}

Navigate using your terminal to the directory where you saved the file. Type in
the following command. It will compile the program into a format that the
computer knows how to execute.

\begin{Terminal}
gcc hello.c
\end{Terminal}

If everything worked, the \ident{gcc} command will exit silently and a new
executable file \file{a.out} will be found in your current directory. If there
was an error, carefully check that each character in \file{hello.c} matches the
C code above and try again.

We can execute the program from the terminal by prefixing \verb|./| to the
filename to tell the terminal to execute a file within the current directory.

\begin{Terminal}
./a.out
Hello, World!
\end{Terminal}

There you have it, your very first C program!

Lets now analyze the anatomy of this simple C program. The very first line,
\verb|#include <stdio.h>| tells the compiler that we would like to use the
\ident{stdio} library. A library is a set of useful, reusable routines. In this
case, the \ident{stdio} library contains routines for interfacing with standard
input and output. Run \verb|man stdio| in your terminal for more information on
the \ident{stdio} library.

The line following our include is simply an empty line. Empty lines do not
effect the program. In some languages, this does matter. However, C is not one
of those languages. You may use empty lines liberally to make your code more
readable.

Next, we have \verb|main ()|. This line tells the compiler that we have a
function named \verb|main|. The \verb|main| function is the beginning of the
program. Every program has one. Lastly, \verb|()| denotes the argument list for
the function. Our \verb|main| function takes no arguments and so it has none
specified\footnote[1]{This isn't technically true, but that will be covered in
a much later chapter.}. Do not worry if this seems mysterious, we will cover it
in great detail later.

After the line defining our function \ident{main}, we have a \verb|{|. This
denotes the beginning of the body of the function. There is a coordinating
\verb|}| at the end of the function. Inside of these curly braces is the crux
of our program.

\verb|printf ("Hello, World!\n");| is a function call to the \ident{printf}
function (part of the \ident{stdio} library). You can tell it is a function
call because the function name is followed by a parenthesized argument list.
In this case, our function call has a single argument,
\verb|"Hello, World!\n"|. Information can be shared between functions in
a couple different ways. This way, through the argument list, is the most
common.

Our \verb|"Hello, World!\n"| argument is what we call a "string". A string
is a sequence of characters. At the end of the string you will see \verb|\n|.
This means that the string should contain a line break at the end of it.

We denote the end of a \ident{statement} in C with a semicolon \ident{;}.
What exactly a \ident{statement} is will be vague for the time being, but it
will become obvious as we continue.

\section{Integers}

Writing a Hello World program is always fun in a new language, but it wont
exactly win us any awards. Lets try to print out some interesting numbers
using our program as a base.

Lets create a new text file with the following and save it as \file{answer.c}.

\begin{code}{answer.c}
#include <stdio.h>

main ()
{
    int answer = 42;
    printf ("The answer is %d, but what is the question?\n", answer);
}
\end{code}

Once again, lets use \ident{gcc} to compile the program. You will get very
used to this in no time!

\begin{Terminal}
gcc answer.c
\end{Terminal}

And again, lets run it.

\begin{Terminal}
./a.out
The answer is 42, but what is the question?
\end{Terminal}

Since we have covered the anatomy of the \ident{main} function earlier in this
chapter, lets skip to the two lines inside the body of the function.

Lets look at the first line, \verb|int answer = 42;|. This line declares a new
variable named \verb|again| of the type \verb|int|. Variables are always
defined in the format \verb|type| followed by \verb|name|. \verb|int| is short
for \verb|integer|.  You might remember that an integer is an "whole" number,
such as 1, 2, or -20.  There are no decimal points in an integer. So if you
tried to store the number \verb|12.5| in an integer, it would simply be 12. The
second half of this statement, \verb|= 42| initialized the variable
\verb|again| to the value \verb|42|.

The second line should look familiar. It again is calling the \verb|printf|
function. However, this time there are two arguments. We will cover this in
more detail in a later chapter, but there are a few important things to take
from this example. First, notice the comma \verb|,| used to separate the
two arguments provided to \verb|printf|. In C, arguments are separated by
commas. Also, note the \verb|%d| inside of the first parameter. This is a
magic value that \verb|printf| will replace with the value of the second
parameter.

\section{Multiplication}

Integers don't seem to be all that useful when used by themselves like this.
Lets take a look how multiplication of integers works with an example. Type
the following code into a new text file named \file{multiply.c}.

\begin{code}{multiply.c}
#include <stdio.h>

main ()
{
    int a = 42;
    int b = 32;

    printf ("%d * %d = %d\n", a, b, a * b);
}
\end{code}

And, again, we compile and run it.

\begin{Terminal}
gcc multiply.c
./a.out
42 * 32 = 1344
\end{Terminal}

In this example, we have two variables defined, \verb|a| and \verb|b|
initialized with the values 42 and 32 respectively. Following that, like before
we call the function \verb|printf|. This time, there are four arguments. You can
see that there are three \verb|%d|s used in the first argument to \verb|printf|.
One for each of the arguments following it. The following two arguments,
unsurprisingly, cause the values 42 and 32 to be placed into the printed line.
However, the last argument, \verb|a * b| performs the multiplication of the
variables \verb|a| and \verb|b| as denoted by the multiplication operator
\verb|*|. The result of this multiplication is placed in the location of the
last \verb|%d| when printing to the screen.

\section{Addition, Subtraction, Division and Modulus}

Multiplication might have been obvious. How about we take a look at a few other
arithmetic operators. See table 1.1 for the list of arithmetic operators.
Operators always follow the \verb|left operator right| pattern. The operator
operates on a value to the left and a value to the right.

So for the multiplication example above, that would be:

\begin{center}
\begin{tabular}{r c l}
left & operator & right \\
\verb|a| & \verb|*| & \verb|b| \\
\end{tabular}
\end{center}

As you can imagine, we can simply substitute the multiplication \verb|*|
operator out for one of the others. The table below gives you our available
arithmetic operators.

\begin{center}
\begin{tabular}{r l}
\hline
\verb|+| & Addition of two numbers \verb|(a + b)| \\
\verb|-| & Subtraction of two numbers \verb|(a - b)| \\
\verb|*| & Multiplication of two numbers \verb|(a * b)| \\
\verb|/| & Division of two numbers \verb|(a / b)| \\
\verb|%| & Modulus (remainder) of integer division \verb|(a % b)| \\
\hline
\end{tabular}
\end{center}

Lets modify the code above to perform division and save it into a new file
called \file{divide.c}.

\begin{code}{divide.c}
#include <stdio.h>

main ()
{
    int a = 221;
    int b = 13;

    printf ("%d / %d = %d\n", a, b, a / b);
}
\end{code}

Now we compile \file{divide.c} and run it (you'll see a pattern here).

\begin{Terminal}
gcc divide.c
./a.out
221 / 13 = 17
\end{Terminal}

Simple, right?

Now I don't think we need to do this for addition and subtraction, but how about
modulus? The modulus \verb|%| operator divides the \verb|left| by the
\verb|right| and returns the remainder. So if we performed the following
division using normal algebra, we would get a fractional result.

\begin{center}
\begin{math}
10 / 3 = 3.3333...
\end{math}
\end{center}

However, what modulus does is try to divide the number as many times as it can
and then return the remainder. Those that have performed division the "long way"
will understand this as the same remainder. So if we were to perform the
modulus, we would get the following.

\begin{center}
\verb|10 % 3 = 10 - 3 - 3 - 3 = 1|
\end{center}

Just to verify our logic, lets modify the previous example in a new file
called \file{modulus.c}.

\begin{code}{modulus.c}
#include <stdio.h>

main ()
{
    int a = 42;
    int b = 32;

    printf ("%d % %d = %d\n", a, b, a % b);
}
\end{code}

Now we compile \file{modulus.c} and run it.

\begin{Terminal}
gcc modulus.c
./a.out
42 % 32 = 10
\end{Terminal}

\section{Student Exercise}

Use what you have learned in this chapter to print the multiplication of
the numbers provided by the user. The \verb|scanf()| function is provided for
you to receive input from the terminal. It will be covered in a later chapter.

\begin{code}{student1.c}
#include <stdio.h>

main ()
{
    int a;
    int b;

    printf ("Enter two numbers to multiply:  ");
    scanf ("%d %d", &a, &b);

    /* TODO: Print the multiplication of a and b. */
}
\end{code}

When complete, the output should look like the following.

\begin{Terminal}
gcc student1.c
./a.out
Enter two numbers to multiply:  10 20
10 * 20 = 200
./a.out
Enter two numbers to multiply:  73 81
73 * 81 = 5913
./a.out
Enter two numbers to multiply:  100 0
100 * 0 = 0
\end{Terminal}


\subsection{Extra Credit}

\begin{enumerate}
\item The type \verb|int| cannot hold every number known. Research the minumum
and maximum values that can be stored in an \verb|int|.
\item Research to discover how many bytes are used in an \verb|int|.
\end{enumerate}
