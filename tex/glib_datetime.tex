\chapter{Dates and Times}

At some point you will need to work with dates and times.
Every operating system has its own quirks in dealing with times.
For this reason, GLib has a couple of abstraction layers dependong on what you need to do.

If what you need is to stick close to the posix \verb|struct tv|, there is \verb|GTimeVal|.
It is also useful if you need to work on a large array of dates and times.
\verb|GDateTime| was added recently to provide a higher-level API this is more convenient in both C and higher level languages.
As one would guess, \verb|GTimeZone| provides information about a timezone such as its GMT offset and if it is daylight savings.

\section{GTimeVal}

\verb|GTimeVal| is a structure that contains two fields.
The first is \verb|tv_sec| and it contains the number of seconds since January 1, 1970.
The second is \verb|tv_usec| and it contains the number of microseconds past the current second.

\marginpar{
    Notice that tv\_sec and tv\_usec are glong.
    Performing arithmetic may overflow on 32-bit architectures sooner than you think!
}
\begin{code}{}
typedef struct
{
    glong tv_sec;
    glong tv_usec;
} GTimeVal;
\end{code}

To get the current time using \verb|GTimeVal|, we use the \verb|g_get_current_time()| function.
It asks the system clock for the current local time and stores it in the structure provided.
You are responsible for that memory.
In this case, it is simply allocated on the stack.
However, it could just as easily be a field in a structure of your own.

\begin{code}{}
GTimeVal tv;
g_get_current_time(&tv);
\end{code}

It is common to covert dates and times to a string for communication with external systems.
ISO-8601 is a commonly supported format for this.
It contains the date, time, and timezone offset and looks like \verb|2012-11-12T23:59:02-0800|.
GLib provides functions to translate to and from this format.

\begin{code}{}
GTimeVal tv;
gchar *str;

g_get_current(&tv);
str = g_time_val_to_iso8601(&tv);
g_time_val_from_iso8601(&tv, str);
g_free(str);
\end{code}

If you find working with a \verb|GTimeVal| too inconvenient you can also use the function \verb|g_get_real_time()|.
It will fetch the number of seconds since the UNIX epoch as a \verb|gint64|.
This is a much better choice if you just need to perform arithmetic on the \verb|tv_sec| field of a \verb|GTimeVal|.


\section{GDateTime}

\verb|GDateTime| is a structure that tracks the combination of a date, time of day, and timezone for you.
One purpose of this structure is to simplify manipulation of dates and times.
For example, with \verb|GTimeVal| you would need to convert to another structure such as \verb|struct tm| to manipulate days, weeks, or months accurately.
With \verb|GDateTime| that is all handled for you.

\verb|GDateTime| is \emph{referenced counted}.
That means that you are responsible for releasing the memory back to the system when you are done with it using \verb|g_date_time_unref()|.

Additionally, \verb|GDateTime| is \emph{immutable}.
That means that every time you make a change, you receive a new instance of a \verb|GDateTime|.
Don't forget to unref them!

\begin{code}{}
GDateTime *now;
GDateTime *then;
gchar *str;

now = g_date_time_new_local();
then = g_date_time_add_weeks(now, 3);
str = g_date_time_printf(then, "%x %X");

g_print("In three weeks, it will be %s\n", str);

g_free(str);
g_date_time_unref(then);
g_date_time_unref(now);
\end{code}

You can event use \verb|GDateTime| as the key in a \verb|GHashTable|.
Simply use the provided \verb|GHashFunc| and \verb|GEqualFunc| for \verb|GDateTime|.

\begin{code}{}
GHashTable *ht;
GDateTime *dt;
gchar *str;

ht = g_hash_table_new_full(g_date_time_hash,
                           g_date_time_equal,
                           g_date_time_unref,
                           g_free);
dt = g_date_time_new_now_local();
str = g_date_time_format(dt, "%x %X");
g_hash_table_insert(ht, dt, str);
\end{code}


\section{GTimeSpan}

\verb|GDateTime| introduces the concept of a time-span as \verb|GTimeSpan|.
This is the number of microseconds between to dates and times.
This is not meant for profiling how long code takes to run.
You would want \verb|GTimer| for that.
However, it is appropriate for determining the amount of time between two chat messages.

\verb|GTimeSpan| is a \verb|gint64|.
Therefore there is no need to free any memory when differencing two \verb|GDateTime|s.
The following example calculates the number of days since January 1, 2000.

\begin{code}{}
GDateTime *now;
GDateTime *jan12000;
GTimeSpan diff;

now = g_date_time_new_now_utc();
jan12000 = g_date_time_new_utc(2000, 1, 1, 0, 0, 0);
diff = g_date_time_difference(jan12000, now);

g_print("%"G_GINT64_FORMAT" days since Jan 1, 2000.\n",
        jan12000 / G_TIME_SPAN_DAY);

g_date_time_unref(now);
g_date_time_unref(jan12000);
\end{code}


\section{GTimeZone}


\section{GTimer}

\verb|GTimer| is a structure used for calculating how much wall clock time has elapsed.
It is typically used to profile how long something takes to run.
For example, if you want to now if a particular optimization is faster, \verb|GTimer| is a very simple way to achieve that.

There is no need to start a timer, it is started as soon as you create it.
Simply stop it when you are finished.
If you want to re-use a timer, you can call \verb|g_timer_reset()| to restart it.

\begin{code}{}
GTimer *timer;
gdouble elapsed;

timer = g_timer_new();
test_1();
g_timer_stop(timer);
elapsed = g_timer_elapsed(timer, NULL);
g_print("%lf seconds elapsed.\n", elapsed);
g_timer_destroy(timer);

\end{code}


\section{Monotonic Time}

Developers often use the \emph{real-time} clock when what they really want is monotonic time.
The real-time clock is what you get from \verb|g_get_current_time()|.
It is the time as humans think of it with dates, times, leap seconds and such.
To illustrate why monotonic time is what we often want, think about the following problem.

If we want to show a dialog in 30 minutes to tell the user to stop using their computer and take a break, how might we go about doing this?
The simple solution is to occasionally check to see what time it is.
If the time period has passed then show the dialog.

What happens if the user crosses a timezone boundry?
Or the clock has been updated because it was incorrect?
You risk the chance of not reaching your 30 minute timeout.

\marginpar{
clock\_gettime() used with CLOCK\_MONOTONIC is what this uses on Linux.
Some systems such as Mac OS X do not support it.
Therefore this can be handy when writing for multiple platforms.
}

The monotonic clock is designed precisely for this purpose.
Instead of giving you a time that is relative to humans, it gives you a time that is relative to the computer.
Often times, this is implemented as the number of seconds since the computer booted.
It is impervious to timezone changes and clock corrections.

\begin{Terminal}
gint64 now = g_get_monotonic_time();
\end{Terminal}

