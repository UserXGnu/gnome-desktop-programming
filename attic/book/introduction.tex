\chapter{Introduction}

Your presence here is important. You have made the choice to remove the magic
curtains from your computing experience by learning the low-level computer
programming language, C. The history of C is long, and can be much better
explained in other mediums. Many changes to the language have occurred over
the past 40 years. This book aims to teach you the C programming language as
it is today, only explaining ancient portions when it provides a better
understanding of the now.

This book covers the C programming language, from the very basics to advanced
concepts. More than that, it is a companion for your journey in Linux and the
GNOME Desktop. Once we have a grasp of the language we will learn the
components available to us and how to use them effectively. We will cover the
concepts behind the choices. We will take note of designs and concepts have
not served us well and learn from them.

The next phase of our journey is one of preparation.


\section{Terminal}

Most of the exercises in this book will involve the use of a terminal. 

A terminal is simply a program which allows a user to interact with a computer
by typing commands and viewing the results.  It provides a no-frills, 
get-things-done interface to the computer's operating system.

Being comfortable using a terminal is an essential skill for a programmer.

Before getting started on the exercises, you will need to be familiar with
basic terminal use, such as changing and listing directories, running commands, 
and interpreting output.

The standard GNOME desktop includes a terminal program called 
\ident{gnome-terminal}, which is what your author is using. \ident{xterm} is
another common, high-quality terminal. 


\section{Text Editor}

Another essential tool for a programmer is a text editor.

There are many text editors for programmers, and every programmer has their
favorite. Some use advanced editors such as VIM or Emacs, both of which have
rather steep learning curves. Many GNOME programmers are perfectly happy in 
Gedit, an editor which is included with the standard GNOME desktop. 

For this book, I will assume you are using Gedit, because the VIM and Emacs
users probably know their way around their editor just fine.

\section{Preparing Your System}

To write software for a particular platform, such as GNOME, you need an SDK
(Software Development Kit). An SDK contains information about the platform. It
typically includes a compiler, headers, and other associated files. Sometimes,
people call this a "toolchain".

A compiler is a program that takes source code and translates it into a
language that the computer knows. If it cannot understand the code then it will
not be able to translate and the compiler will attempt to give us a hint as to
what went awry.

"Headers" contain information about the programs on our computer.  It gives us
insight into how the programs were written. When we want to make use of
existing software on the system, we use that software's "headers" to prime the
compiler as to how it works. The compiler than then takes our source code and
with what it learned from the headers to translate our program into something
the machine can execute.


\section{Installing the SDK}

GNOME contains many pieces of software written by many different groups of
people. We need a compiler, headers, and associated tools for building
software. To get those, we will use our Linux distributions package manager to
install the required components.


\subsection{Fedora 17}

\begin{Terminal}
sudo yum install gtk3-devel clutter-devel gobject-introspection-devel webkitgtk3-devel glibc-devel make automake autoconf pkg-config gedit
\end{Terminal}


\subsection{Ubuntu 12.04}

\begin{Terminal}
sudo apt-get install build-essential libgtk-3-dev pkg-config gedit
\end{Terminal}


\subsection{Arch Linux}
\begin{Terminal}
sudo pacman -S base-devel clutter-gtk gtk gedit
\end{Terminal}

\subsection{Debian Wheezy}

\begin{Terminal}
sudo apt-get install build-essential libgtk3.0-dev pkg-config gedit
\end{Terminal}


\section{Kicking the Tires}

Let's double-check that everything has been installed properly so that 
unexpected problems don't surface when we perform the exercises in this book.
We will use the \ident{pkg-config} program to check what versions of programs
we've just installed. \ident{pkg-config} is a program that tells us information
about various components of our installed SDK.

\begin{Terminal}
$ pkg-config --modversion gobject-2.0
2.33.14

$ pkg-config --modversion gtk+-3.0
3.5.18
\end{Terminal}

If the above commands give you errors about missing packages, then you still
need to install the development (-dev) packages for the given libraries.

Your author remembers how rewarding it was to see his first Gtk window display
on the screen. To share that feeling with you, we will start with a very
ambitious first program. You likely wont know what the various parts of the
following code is, but we will get to that in the following chapters. This
example is about getting a taste of the future!

Create a new text file named \file{hellogtk.c} containing the following code.
Ignore the line numbers on the left of the margin.

\begin{code}{hellogtk.c}
#include <gtk/gtk.h>

gint
main (gint   argc,
      gchar *argv[])
{
    GtkWidget *label;
    GtkWidget *window;

    gtk_init (&argc, &argv);

    window = gtk_window_new (GTK_WINDOW_TOPLEVEL);
    label = gtk_label_new ("Hello, Gtk World!");
    gtk_container_add (GTK_CONTAINER (window), label);
    gtk_widget_show (label);
    g_signal_connect (window, "delete-event", gtk_main_quit, NULL);
    gtk_window_present (GTK_WINDOW (window));

    gtk_main ();

    return 0;
}
\end{code}

Navigate using your terminal to the directory containing the file you just
created and run the following commands.

\begin{Terminal}
cc hellogtk.c $(pkg-config --cflags --libs gtk+-3.0)
./a.out
\end{Terminal}

Congratulations on your first Gtk+ program!
