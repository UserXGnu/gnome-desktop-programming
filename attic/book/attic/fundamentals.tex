\chapter{Fundamentals}

This chapter will familiarize you with the syntax and terminology of the C
language, as well as the process of reading and writing C code.

Understanding the fundamentals of C is critical to your success as a GNOME
developer. If you have a strong understanding of the basics, no challenge will 
be unsurmountable with the right amount of patience and determination.

\section{Goals}

By completing this chapter you will learn the skills to perform the following
tasks:

\begin{enumerate}
\item Point out the following parts of a program: inclusion of header, function
arguments and body, variable type and name, strings and integers, and
arithmetic operators.
\item Write a basic C program with proper syntax that can multiply and divide
integers.
\end{enumerate}


Today, you start on those fundamentals. Let's get things going by looking at
some C code and familiarizing ourselves with its anatomy.


\begin{code}{figure1.c}
\input{fundamentals1}
\end{code}

TODO
\begin{enumerate}
\item \verb|#include <glib.h>|
\item \verb|gint|
\item \verb|(gint argc, gchar *argv[])|
\item \verb|{|
\item \verb|    gint age;|
\item \verb|    return 0;|
\item \verb|}|
\end{enumerate}

TODO: The whole following is not easy to understand or deduce, but it
starts covering the sections, which can be elaborated on.

Lets look at the first line: \verb|#include <glib.h>|. This line includes
the \ident{header} \verb|glib.h|. A \ident{header} is a file that describes
various components of C code. \verb|glib.h|, in particular, describes many
of the core components we will use on the GNOME desktop.

TODO: More info on headers. Maybe we should save it for later?

The next line of code: \verb|gint| contains the \ident{return type} for the
function called main.

Following the type we have \verb|main|, the name of the function that is
called when the program starts.

TODO: what does "calling a function" even mean at this point?



TODO: Lots of explaination on the anatomy of figure 1.


\section{Arithmetic}

Every computer language, at some point, is working with numbers and basic
arithmetic. Thankfully, you don't have to be good at mathematics to be a good
programmer. But we should have a knowledge of the basics. Performing arithmetic
in the C programming language should feel similar to what you already do when
writing equations on a piece of paper.

Addition is performed with the \ident{+} operator and subtraction with the
\ident{-} operator. Multiplication is performed with the \ident{*} operator and
division with the \ident{/} operator. Determining the remainder of division can
be found with the \ident{\%}, "modulus", operator.

TODO: examples of the arithmetic.

\subsection{Order of Operations}

Just like you may have learned in beginning algebra, parenthesis can be used
to alter the order of operations.

TODO: explain the order of operations.

\section{Numeric Types}

\begin{tabular}{ r | c | c | l }
\hline
Type & Signed & Size in Bits & Description \\
\hline
\verb|gint8| & Yes & 8 &
An 8-bit, signed, integer that can \\
&&& hold a minimum value of -128 and a \\
&&& maximum value of 127. \\
\hline
\verb|gint16| & Yes & 16 &
A 16-bit, signed, integer that can \\
&&& hold a minimum value of -32,768 and a \\
&&& maximum value of 32,767. \\
\hline
\verb|gint32| & Yes & 32 &
A 32-bit, signed, integer that can \\
&&& hold a minimum value of -2,147,483,648 \\
&&& and a maximum value of 2,147,483,647. \\
\hline
\verb|gint64| & Yes & 64 &
A 64-bit, signed, integer that can \\
&&& hold a minimum value of \\
&&& -9,223,372,036,854,775,808 and a \\
&&& maximum value of \\
&&& 9,223,372,036,854,775,807. \\
\hline
\verb|guint8| & No & 8 &
An 8-bit, unsigned, integer that can \\
&&& hold a minimum value of 0 and a \\
&&& maximum value of 255. \\
\hline
\verb|guint16| & No & 16 &
A 16-bit, unsigned, integer that can \\
&&& hold a minimum value of 0 and a \\
&&& maximum value of 65535. \\
\hline
\verb|guint32| & No & 32 &
A 16-bit, unsigned, integer that can \\
&&& hold a minimum value of 0 and a \\
&&& maximum value of 4,294,967,295. \\
\hline
\verb|guint64| & No & 64 &
A 16-bit, unsigned, integer that can \\
&&& hold a minimum value of 0 and a \\
&&& maximum value of \\
&&& 18,446,744,073,709,551,615. \\
\hline
\verb|gfloat| & - & 32 & \\
\hline
\verb|gdouble| & - & 64 & \\
\hline
\end{tabular}

\section{Binary Counting}

\section{Types}


\section{Student Exercises}

Complete the following exercises. Each exercise is designed to enforce the
lessons described in this chapter.


\subsection{Exercise 1}
Complete the following source code to calculate how old the user will be this
year based on the year they were born. Use the constant \verb|THIS\_YEAR| in
the arithmetic and store the result in \verb|now|.

\begin{code}{exercise1.c}
\input{exercise1}
\end{code}
\footnotesize
\input{exercise1.out}
\rule{\textwidth}{0.1pt}


\subsection{Exercise 2}
Complete the following source code to perform basic division on a number
provided by a user.

\begin{code}{exercise2.c}
\input{exercise2}
\end{code}
\footnotesize
\input{exercise2.out}
\rule{\textwidth}{0.1pt}

\subsubsection{Questions}

\begin{enumerate}
\item What is the difference between \verb|(numerator/9)| and
      \verb|(numberator/9.0)|?
\end{enumerate}
