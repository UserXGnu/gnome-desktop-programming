% GnomeUniversity - 101 - Chapter 1
\documentclass[12pt]{article}
\usepackage[
    pdftitle={Introduction to C - Chapter 1},
    pdfauthor={Christian Hergert},
    pdfsubject={GNOME University},
    pdfkeywords={Programming},
    bookmarks, bookmarksopen,
    pdfstartview=FitH,
    colorlinks,linkcolor=blue,citecolor=blue,
    urlcolor=red,
]{hyperref}
\usepackage{color}
\usepackage{listings}
\usepackage{parskip}
\usepackage{graphicx}
\usepackage{tikz}
\parskip 7.2pt

\newcommand{\versal}[1]{\noindent{\Huge #1\kern-.10em}}
\newcommand{\file}[1]{{\bf\ttfamily #1}}
\newcommand{\ident}[1]{{\it\ttfamily #1}}
\newcommand{\shell}[1]{{\it\ttfamily #1}}
\newcommand{\python}[1]{{\it\ttfamily #1}}
\newcommand{\ruby}[1]{{\it\ttfamily #1}}
\newcommand{\book}[2]{{\it\ttfamily #1} by {\it #2}}
\newcommand{\program}[1]{\it\ttfamily #1}

\lstnewenvironment{code}[1]
{
    \lstset{language=C,
            showstringspaces=false,
            basicstyle=\ttfamily\small,
            caption=#1,
            numbers=left}
}
{}

\lstnewenvironment{Terminal}
{
    \lstset{basicstyle=\ttfamily,
            breaklines=true}
}
{}

\title{GNOME University \\
Introduction to C \\
Chapter 1}
\author{Christian Hergert \\
    \small \texttt{christian@hergert.me}
}

\date{October 2012}

\begin{document}
\maketitle

Welcome to GNOME University! This chapter will get you started down the path of
programming in C by introducting the process of writing source code in C,
compiling, and executing the resulting program. If you don't know what any of
that means, that's okay, we will learn!

\section{Source Code}

To write programs using the C programming language, we begin with an empty text
file. In that file, we write a sequence of characters and symbols to construct
the program as we like. After which, we use a tool called a compiler to convert
that source code into machine code. Machine code is a representation of your
program in a format that your computers CPU can execute. As you might imagine,
diferent types of CPUs have different representaions of your program. Compilers
allow us to convert a single piece of code into machine code for multiple types
of CPUs.

\section{Hello, World}

Let's start by starting up our text editor of choice. If you don't yet have a
favorite text editor, try \verb|gedit|. It is available as part of your GNOME
desktop installation.

Copy the following listing into your text editor. Be careful and methodical as
you go. When copying, notice the characters and keywords used.  You will become
familiar with them. When I learn a new language, I like to guess and reason
about what the language does before I actually learn it.

\textbf{Do not copy the line numbers} to the left of the margin. They are
provided for readability within the chapter text.

\begin{code}{hello.c}
#include <stdio.h>

int
main (int   argc,
      char *argv[])
{
    printf ("Hello, World!\n");
    return 0;
}
\end{code}

Save the source code as a new file named \file{hello.c}. The next step in our
process is to translate the program into code that the computer knows how to
execute. This is called compiling. We will use the program \verb|gcc| to
perform this task.

Open a terminal and navigate to the directory where you saved \file{hello.c}.
Compile the source code into a program using the following command.

\begin{Terminal}
gcc -Wall -Werror -o hello hello.c
\end{Terminal}

If everything worked, the \verb|gcc| command will exit silently and a new
executable file named \file{hello} will have been placed in your current
directory. If instead there was an error, carefully check that each character in
\file{hello.c} matches the C code above and try again.

Once you have successfully compiled \file{hello.c}, we can execute the program
from the terminal by prefixing \verb|./| to the filename to tell the terminal
to execute a file within the current directory.

\begin{Terminal}
./hello
Hello, World!
\end{Terminal}

There you have it, your very first C program!

\section{Anatomy}

Lets now analyze the anatomy of this simple C program. The very first line,
\verb|#include <stdio.h>| tells the compiler that we would like to use the
\ident{stdio} library. A \textbf{library} is a set of useful, reusable
routines. In this case, the \ident{stdio} library contains routines to
interface with the standard input and output of your terminal\footnote{Run
\ttfamily{man stdio} \rmfamily in your terminal for more information on the
stdio library.}.  This allows your program to output text to the terminal and
input text from it.

The line following \verb|#include| is simply an empty line. Empty lines do
not affect the program. In some languages, this does matter. However, C is not
one of those languages. You may use empty lines liberally to make your code
more readable.

Next, we have \verb|main (int argc, char *argv[]))|. This tells the compiler
that we have a function named \verb|main|. The \verb|main| function is the
beginning of the program.  Every program has one. Every program gets two
arguments, \verb|argc| and \verb|argv|. Arguments are declared inside of the
pair of parenthesis \verb|()|. We will go into what these parameters mean in a
later chapter. Do not worry if this seems mysterious, we will cover it in detail
in a later chapter.

After the line defining our function \ident{main}, we have a \verb|{|. This
denotes the beginning of the body of the function. There is a corresponding
\verb|}| at the end of the function. Inside of these curly braces is the crux
of our program.

\verb|printf ("Hello, World!\n")| is a function call to the \ident{printf}
function (part of the \ident{stdio} library). You can tell it is a function call
because the function name is followed by a parenthesized argument list.  In this
case, our function call has a single argument, \verb|"Hello, World!\n"|.
Information can be shared between functions in a couple ways. This way, through
the argument list, is the most common.

Our \verb|"Hello, World!\n"| argument is what we call a "string". A string is a
sequence of characters contained within two quotation marks (\verb|"|).  At the
end of this string you see \verb|\n|.  This means that the string should contain
a line break at the end of it.

We denote the end of a \ident{statement} in C with a semicolon (\ident{;}).
What exactly a \ident{statement} is will be vague for now, but it will become
clear as we continue.

\section{Integers}

Writing a Hello World program is always fun in a new language, but it wont
exactly win us any awards. Lets try to print out some interesting numbers
using our program as a base.

Lets create a new text file with the following and save it as \file{answer.c}.

\begin{code}{answer.c}
#include <stdio.h>

int
main (int   argc,
      char *argv[])
{
    int a = 42;
    printf ("Answer is %d, what's the question?\n", a);
    return 0;
}
\end{code}

Once again, let's use \verb|gcc| to compile the program. You will get very used
to this in no time!

\begin{Terminal}
gcc -Wall -Werror -o answer answer.c
\end{Terminal}

And again, let's run it.

\begin{Terminal}
./answer
Answer is 42, what's the question?
\end{Terminal}

Since we have covered the anatomy of the \ident{main} function earlier in this
chapter, lets skip to the three lines inside the body of the function. Remember
that these are the lines inside of the curly braces \verb|{| and \verb|}|.

\begin{figure}
	\centering
	\input{chapter1_fig1}
	\caption{Variable assignment}
\end{figure}

Lets look at the first line, \verb|int a = 42;|. This line declares a new
variable named \verb|a| of the type \verb|int|. Variables are always
defined in the format \verb|type| followed by \verb|name|. \verb|int| is short
for \verb|integer|.  You might remember that an integer is an "whole" number,
such as 1, 2, or -20.  There are no decimal points in an integer. So if you
tried to store the number \verb|12.5| in an integer, it would simply be 12. The
second half of this statement, \verb|= 42| initialized the variable
\verb|a| to the value \verb|42|.

The second line should look familiar. It again is calling the \verb|printf|
function. However, this time there are two arguments. We will cover this in
more detail in a later chapter, but there are a few important things to take
from this example. First, notice the comma \verb|,| used to separate the
two arguments provided to \verb|printf|. In C, arguments are separated by
commas. Also, note the \verb|%d| inside of the first parameter. This is a
magic value that \verb|printf| will replace with the value of the second
parameter.

\end{document}
